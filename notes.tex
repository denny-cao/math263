%++++++++++++++++++++++++++++++++++++++++
\documentclass[article, 12pt]{article}
\usepackage{float}
\usepackage{setspace}
\usepackage{tabu} % extra features for tabular environment
\usepackage{amsmath}  % improve math presentation
\usepackage{graphicx} % takes care of graphic including machinery
\usepackage[margin=1in]{geometry} % decreases margins
\usepackage{cite} % takes care of citations
\usepackage[final]{hyperref} % adds hyper links inside the generated pdf file
\usepackage{tikz}
\usepackage{caption} 
\usepackage{fancyhdr}
\usepackage{amssymb} % symbols like /therefore
\usepackage{amsthm} % proofs
\usepackage{enumerate} % lettered lists
\usepackage{mathtools} % macros
\usetikzlibrary{scopes}
% \usepackage{xcolor} \pagecolor[rgb]{0.12549019607,0.1294117647,0.13725490196} \color[rgb]{0.82352941176,0.76862745098,0.62745098039} % dark theme
\theoremstyle{definition}
\newtheorem{example}{Example}[subsection]
\newtheorem*{remark}{Remark}
\newtheorem{theorem}{Theorem}[subsection]
\newtheorem{definition}{Definition}[subsection]
\newtheorem{corollary}{Corollary}[subsection]
\hypersetup{
	colorlinks=false,      % false: boxed links; true: colored links
	linkcolor=blue,        % color of internal links
	citecolor=blue,        % color of links to bibliography
	filecolor=magenta,     % color of file links
	urlcolor=blue         
}
\usepackage{physics}
\usepackage{siunitx}
\usepackage{tikz,pgfplots}
\usepackage[outline]{contour} % glow around text
\usetikzlibrary{calc}
\usetikzlibrary{angles,quotes} % for pic
\usetikzlibrary{arrows.meta}
\tikzset{>=latex} % for LaTeX arrow head
\contourlength{1.2pt}

\colorlet{xcol}{blue!70!black}
\colorlet{vcol}{green!60!black}
\colorlet{myred}{red!70!black}
\colorlet{myblue}{blue!70!black}
\colorlet{mygreen}{green!70!black}
\colorlet{mydarkred}{myred!70!black}
\colorlet{mydarkblue}{myblue!60!black}
\colorlet{mydarkgreen}{mygreen!60!black}
\colorlet{acol}{red!50!blue!80!black!80}
\tikzstyle{CM}=[red!40!black,fill=red!80!black!80]
\tikzstyle{xline}=[xcol,thick,smooth]
\tikzstyle{mass}=[line width=0.6,red!30!black,fill=red!40!black!10,rounded corners=1,
                  top color=red!40!black!20,bottom color=red!40!black!10,shading angle=20]
\tikzstyle{faded mass}=[dashed,line width=0.1,red!30!black!40,fill=red!40!black!10,rounded corners=1,
                        top color=red!40!black!10,bottom color=red!40!black!10,shading angle=20]
\tikzstyle{rope}=[brown!70!black,very thick,line cap=round]
\def\rope#1{ \draw[black,line width=1.4] #1; \draw[rope,line width=1.1] #1; }
\tikzstyle{force}=[->,myred,very thick,line cap=round]
\tikzstyle{velocity}=[->,vcol,very thick,line cap=round]
\tikzstyle{Fproj}=[force,myred!40]
\tikzstyle{myarr}=[-{Latex[length=3,width=2]},thin]
\def\tick#1#2{\draw[thick] (#1)++(#2:0.12) --++ (#2-180:0.24)}
\DeclareMathOperator{\sn}{sn}
\DeclareMathOperator{\cn}{cn}
\DeclareMathOperator{\dn}{dn}
\def\N{80} % number of samples in plots


\usepackage{titling}
\renewcommand\maketitlehooka{\null\mbox{}\vfill}
\renewcommand\maketitlehookd{\vfill\null}
\usepackage{siunitx} % units
\usepackage{verbatim} 
\newcommand{\courseNumber}{MATH 273}
\newcommand{\courseName}{Discrete Mathematics 2}
\newcommand{\professor}{Dr. Petrescu}
\newcommand{\name}{Denny Cao}
\pagestyle{fancy}
\fancyhf{}% clears all header and footer fields
\fancyfoot[C]{--~\thepage~--}
\renewcommand*{\headrulewidth}{0.4pt}
\renewcommand*{\footrulewidth}{0pt}
\lhead{\name}
\chead{\courseNumber: \courseName}
\rhead{\professor}


\fancypagestyle{plain}{%
  \fancyhf{}% clears all header and footer fields
  \fancyfoot[C]{--~\thepage~--}%
  \renewcommand*{\headrulewidth}{0pt}%
  \renewcommand*{\footrulewidth}{0pt}%
}

% Shortcuts
\DeclarePairedDelimiter\ceil{\lceil}{\rceil} % ceil function
\DeclarePairedDelimiter\floor{\lfloor}{\rfloor} % floor function

\DeclarePairedDelimiter\paren{(}{)} % parenthesis

\newcommand{\df}{\displaystyle\frac} % displaystyle fraction
\newcommand{\qeq}{\overset{?}{=}} % questionable equality

\newcommand{\Mod}[1]{\;\mathrm{mod}\; #1} % modulo operator

% Sets
\DeclarePairedDelimiter\set{\{}{\}}
\newcommand{\unite}{\cup}
\newcommand{\inter}{\cap}

\newcommand{\reals}{\mathbb{R}} % real numbers: textbook is Z^+ and 0
\newcommand{\ints}{\mathbb{Z}}
\newcommand{\nats}{\mathbb{N}}
\newcommand{\rats}{\mathbb{Q}}

\newcommand{\degree}{^\circ}

% Div Operator
\makeatletter
\newcommand*{\bdiv}{%
  \nonscript\mskip-\medmuskip\mkern5mu%
  \mathbin{\operator@font div}\penalty900\mkern5mu%
  \nonscript\mskip-\medmuskip
}
\makeatother

\newcommand{\comp}{\circ} % composite

% Counting
\newcommand\perm[2][^n]{\prescript{#1\mkern-2.5mu}{}P_{#2}}
\newcommand\comb[2][^n]{\prescript{#1\mkern-0.5mu}{}C_{#2}}

\setlength\parindent{0pt}

% Sign Charts
\newdimen\tcolw \tcolw=2.5em % the column width
\edef\ecatcode{\catcode`&=\the\catcode`&\relax}\catcode`&=4
\def\sgchart#1#2{\vbox{\offinterlineskip\halign{\hfil##\quad&##\hfil\crcr\sgchartA#2,:,%
   \omit\sgchartR&\kern.2pt\sgchartS{.5\tcolw}\relax\sgchartE#1,\relax,%
   \sgchartS{.5\tcolw}\relax\cr
   \noalign{\kern2pt}&\def~{}\kern.5\tcolw\sgchartD#1,\relax,\cr}}}
\def\sgchartA#1:#2,{\cr\ifx,#1,\else $#1$&\sgchartB#2{}\expandafter\sgchartA\fi}
\def\sgchartB#1{\hbox to\tcolw{\hss$#1$\hss}\sgchartC}
\def\sgchartC#1{\ifx,#1,\else
   \strut\vrule\kern-.4pt\hbox to\tcolw{\hss$#1$\hss}\expandafter\sgchartC\fi}
\def\sgchartD#1#2,{\ifx\relax#1\else\hbox to\tcolw{\hss$#1#2$\hss}\expandafter\sgchartD\fi}
\def\sgchartE#1#2,{\ifx\relax#1\else
    \ifx~#1\sgchartS\tcolw\circ \else\sgchartS\tcolw\bullet\fi \expandafter\sgchartE\fi}
\def\sgchartR{\leaders\vrule height2.8pt depth-2.4pt\hfil}
\def\sgchartS#1#2{\hbox to#1{\kern-.2pt\sgchartR \ifx\relax#2\else
   \kern-.7pt$#2$\kern-.7pt\sgchartR\fi\kern-.2pt}}
\ecatcode
%++++++++++++++++++++++++++++++++++++++++
\title{
    \vspace{2in}
    \textmd{\textbf{\courseNumber: \courseName}}
    \normalsize\vspace{0.1in}\\
    \vspace{0.1in}\large{\text{\professor}}
    \vspace{3in}
}

\author{\name}
\date{Final: April 26, 2023}

\begin{document}
    \maketitle
    \thispagestyle{empty}
    \pagebreak
    \tableofcontents
    \pagebreak

    \section{Relations}
    \subsection{Introduction}
    \subsubsection{Cartesian Products}
    \begin{definition}
    \label{def:cartesian product}
        Let $A$ and $B$ be sets. The \textbf{cartesian product} of $A$ and $B$ is the set
        \begin{equation}
            A \times B = \set*{(a, b) \mid a \in A \land b \in B}
        \end{equation}
    \end{definition}
    \begin{itemize}
        \item $A \times B \neq B \times A$
        \item Recall there are $2^{|S|}$ subsets of $S$. These are the amount of relations from $A$ to $B$ (A subset of the cartesian product is a relation). Remember that this includes $\emptyset$.
        \item Every function is a relation, but not every relation is a function. When it is a function, it is one-to-one.
    \end{itemize}

    \begin{example}
        Let $A = \set*{0,1,2}$ and $B=\set*{a,b}$. Then $\set*{(0,a),(0,b),(1,a),(2,b)}$ is a relation from $A$ to $B$. This means, for instance, that $0Ra$, but that $1 \not R b$ Relations can be represented graphically. Another way is to use a table:
        \begin{figure}
            \centering
            \begin{tikzpicture}

            \end{tikzpicture}
        \end{figure}
        \begin{figure}[H]
            \centering
            \begin{tabular}{c|c c}
                R & a & b \\
                \hline
                0 & 1 & 1 \\
                1 & 1 & 0 \\
                2 & 0 & 1 \\
            \end{tabular}
        \end{figure}
    \end{example}
    \begin{definition}
        A \textbf{relation on a set} $A$ is a relation from $A$ to $A$
    \end{definition}
    \begin{example}
        Let $A$ be the set $\set*{1,2,3,4}$. Which ordered pairs are in the relation $R = \set*{(a,b) \mid a \bdiv b}$?
        \begin{figure}[H]
            \centering
            \begin{tabular}{c|c c c c}
                R & 1 & 2 & 3 & 4 \\
                \hline
                1 & 1 & 1 & 1 & 1 \\
                2 & 0 & 1 & 0 & 1 \\
                3 & 0 & 0 & 1 & 0 \\
                4 & 0 & 0 & 0 & 1 \\
            \end{tabular}
            \label{fig:example 1.1.2}
        \end{figure}
        \begin{itemize}
            \item Note that this can be a matrix!
        \end{itemize}
    \end{example}
    \subsection{Properties of Relations}
    \begin{definition}
    \label{def:reflexive}
        A relation $R$ on a set $A$ is called \textbf{reflexive} if
        \begin{equation}
            \forall a \in A, (a,a) \in R
        \end{equation}
    \end{definition}
    \begin{itemize}
        \item If the relation is reflexive, then \textbf{the main diagonal of the matrix is full}
    \end{itemize}
    \begin{definition}
    \label{def:symmetric and antisymmetric}
        A relation $R$ on a set $A$ is called \textbf{symmetric} if:
        \begin{equation}
            \forall a \forall b \in A, (a,b) \in R \implies (b,a) \in R
        \end{equation}
        A relation $R$ on a set $A$ is called \textbf{antisymmetric} if:
        \begin{equation}
            \forall a \forall b \in A, (a,b) \in R \implies (b,a) \in R
        \end{equation}
    \end{definition}
    \begin{itemize}
        \item If the relation is symmetric, then \textbf{the matrix is symmetric}. This means that $A = A^T$, and can be seen if the upper and lower triangles are the same.
        \item If the relation is antisymmetric, then \textbf{the matrix is antisymmetric} (The main diagonal is empty)
    \end{itemize} 
    \begin{definition}
    \label{def:transitive}
        A relation $R$ on a set $A$ is called \textbf{transitive} if:
         \begin{equation}
            \forall a \forall b \forall c \in A, (a,b) \in R \land (b,c) \in R \implies (a,c) \in R.
         \end{equation}
    \end{definition}
    \subsection{Combining Relations}
    Similar to composing functions. We can combine relations in any way two sets can be combined. You can do everything you can do with sets. 
    \begin{definition}
        Let $R$ be a relation from a set $A$ to a set $B$ and $S$ a relation from $B$ to a set $C$. The \textbf{composite} of $R$ and $S$ is the relation consisting of ordered pairs $(a,c)$, where $a \in A, c \in C$, and for which $\exists b \in B \mid (a,b) \in R \land (b,c) \in S$. We denote the composite of $R$ and $S$ by $S \comp R$.
    \end{definition}
    \begin{itemize}
        \item If there is an $A$ in $R$ that maps to $B$, and for $S$ there is a $B$ that maps to a $C$, then the composite $S \comp R$ will map $A$ to $C$.
        \item If $B$ maps to multiple $C$'s?, then the composite will map $A$ to multiple $C$'s.
    \end{itemize}
    \begin{definition}
    \label{def:reflexive powers}
        Let $R$ be a relation on the set $A$. The powers $R^n$, $n=1,2,3,\dots$ are defined recursively by:
        \begin{equation}
            R^1 = R \quad \text{and} \quad R^{n+1} = R \comp R^n
        \end{equation}
    \end{definition}
    \begin{definition}
        The relation $R$ on a set $A$ is \textbf{transitive} if and only if: 
    \begin{equation}
        \forall n \geq 1, R^n \subseteq R
    \end{equation}
    \end{definition}
\end{document}
